Since this method does not require any parameters, and is very light it is suitable for real-time applications.
One issue however is that it is slower than methods such as MLS. 
This is due to the overhead in python and lack of concurrency.
Thus, one potential future work is to implement this method in C++ to reduce the overhead since in this work the bilateral filter was already implemented in C++.
Further, this method can benefit greatly from concurrency.
There are numerous works that parallelize the creation of octree structures \cite{paralell-octree}. 
Using an octree one can also parallelize the bilateral filter as well as in \cite{3d_bilateral_filter_ipol}.
With both of these optimizations, the method can be made faster and more suitable for real-time applications.
%Further, with works such as PVCNN \cite{pvcnn} the use of voxelization can also potentially be used to improve existing deep methods. 
