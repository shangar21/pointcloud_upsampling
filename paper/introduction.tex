A pointcloud is an unordered 3D representation of a set of points in space.
It is commonly used in computer graphics, computer vision, and robotics.
Pointclouds are often generated using 3D scanners, LIDAR, or photogrammetry.

In this work we wish to upsample a pointcloud. Given a set of point clouds, we wish to find a new set of points that are more dense but still represent the same underlying surface.
The unstructured and unordered nature of point clouds makes this a challenging problem. Further, the new points while preserving the underlying structure should not introduce any new artifacts, and should be informative and not clustered around the original points.
Further, existing methods for point cloud upsampling often are computationally expensive and require extensive training and parameter tuning. 

To address the above challenges we present a data-structure-driven method for point cloud upsampling that is fast and parameter free. 
Our method utilizes an octree data structure without a depth limit to understand the underlying structure and initially add points to the empty children of the octree.
The lack of a depth limit allows tighter fitting bounding cubes around points and allows for a more accurate representation of the underlying structure.
This representation is often noisy and coarse but captures the underlying structure of the point cloud. 
To smooth the point cloud, we use a bilateral filter in a point cloud application \cite{3d_bilateral_filter_ipol} to smooth the point cloud.

Point cloud upsampling can be used as a downstream task for various applications such as 3D reconstruction, 3D object recognition, and 3D rendering. 
It can be used to improve the quality of surface reconstruction, enhance object detection, extract features more accurately, and more. 

Our method, namely BOLT, learns the geometry and strucutre of the point cloud, upsamples and smooths it without any parameters, deep learning, or fine tuning.
