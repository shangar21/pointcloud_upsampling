Experiments were done mostly with the ShapeNet dataset \cite{shapenet}, which is a large dataset of 3D models. A random set of either 1024 or 2048 point clouds were sampled and then upscaled to double or quadruple the number of points. 

\subsection{Evaluation Metrics}

We evaluate our model using the Chamfer distance, Hausdroff distance and point-to-surface distance.

The Chamfer distance is a measure of how different 2 shapes are and is defined as the following:

$$ C(P, Q) = \dfrac{1}{|P|} \sum\limits_{p \in P} \min_{q \in Q} \norm{p - q}^2 +  \dfrac{1}{|Q|} \sum\limits_{q \in Q} \min_{p \in P} \norm{p - q}^2 \label{eq:chamfer}$$

The Hausdroff distance is a measure of how similar 2 sets are. It is defined as the following:

$$ H(A, B) = \max(h(A, B), h(B, A))\label{eq:hausdroff}$$

Where:

$$h(A, B) = \max_{a \in A} \min_{b \in B} \norm{a - b}$$

\subsection{Without Smoothing}

In this subsection we compare results with octree upsampling only, and no smoothing. 
Qualitatively, the results are not as good as the results with smoothing, and do not fair well
against other methods.

\begin{figure*}[h]
	\centering
	\begin{subfigure}{0.3\textwidth}
		\centering
		\includegraphics[width=\textwidth]{1024_no_upsampling.png}
		\caption{Original point cloud, with 1024 points}
	\end{subfigure}
	\begin{subfigure}{0.3\textwidth}
		\centering
		\includegraphics[width=0.65\textwidth]{./2048_octree_no_smoothing.png}
		\caption{Upsampled point cloud, with 2048 points using only octree and no smoothing}
	\end{subfigure}
	\begin{subfigure}{0.3\textwidth}
		\centering
		\includegraphics[width=0.65\textwidth]{./4096_octree_no_smoothing.png}
		\caption{Upsampled point cloud, with 4096 points using only octree and no smoothing}
	\end{subfigure}	
	\caption{A comparison of the original point cloud and the upsampled point cloud using only octree and no smoothing.}
	\label{fig:no_smoothing}
\end{figure*}

\subsection{With Smoothing}

In this subsection we compare results with octree upsampling and smoothing.

\begin{figure*}[h]
	\centering
	\begin{subfigure}{0.3\textwidth}
		\centering
		\includegraphics[width=\textwidth]{1024_no_upsampling.png}
		\caption{Original point cloud, with 1024 points}
	\end{subfigure}
	\begin{subfigure}{0.3\textwidth}
		\centering
		\includegraphics[width=0.65\textwidth]{./2048_octree_bilateral}
		\caption{Upsampled point cloud, with 2048 points using only octree and bilateral smoothing}
	\end{subfigure}
	\begin{subfigure}{0.3\textwidth}
		\centering
		\includegraphics[width=0.65\textwidth]{./4096_octree_bilateral.png}
		\caption{Upsampled point cloud, with 4096 points using only octree and no smoothing}
	\end{subfigure}	
	\caption{A comparison of the original point cloud and the upsampled point cloud using only octree and no smoothing.}
	\label{fig:no_smoothing}
\end{figure*}
