In this appendix, we provide additional details on the experiments conducted in this paper.
We also provide some qualitative results of our method compared to the other methods used in this paper.
To set up the rest, the ground truths are provided in the following figure:

\begin{figure}[H]
	\centering
	\begin{subfigure}{0.18\textwidth}
		\includegraphics[width=\textwidth]{./images/ground_truths/plane.png}
		\caption{Plane}
	\end{subfigure}
	\begin{subfigure}{0.18\textwidth}
		\includegraphics[width=\textwidth]{./images/ground_truths/helmet.png}
		\caption{Helmet}
	\end{subfigure}
	\begin{subfigure}{0.18\textwidth}
		\includegraphics[width=\textwidth]{./images/ground_truths/cap.png}
		\caption{Cap}
	\end{subfigure}
	\begin{subfigure}{0.18\textwidth}
		\includegraphics[width=\textwidth]{./images/ground_truths/car.png}
		\caption{Car}
	\end{subfigure}
	\begin{subfigure}{0.18\textwidth}
		\includegraphics[width=\textwidth]{./images/ground_truths/headset.png}
		\caption{Headset}
	\end{subfigure}
\end{figure}

\subsection{Comparison of Different smoothing methods}

In this subsection we qualitatively compare bilateral filtering with KNN filtering and no filtering.

Bilateral filtering:


\begin{figure}[H]
	\centering
	\begin{subfigure}{0.18\textwidth}
		\includegraphics[width=\textwidth]{./images/bolt/plane.png}
		\caption{BOLT plane}
	\end{subfigure}
	\begin{subfigure}{0.18\textwidth}
		\includegraphics[width=\textwidth]{./images/bolt/helmet.png}
		\caption{BOLT helmet}
	\end{subfigure}
	\begin{subfigure}{0.18\textwidth}
		\includegraphics[width=\textwidth]{./images/bolt/cap.png}
		\caption{BOLT cap}
	\end{subfigure}
	\begin{subfigure}{0.18\textwidth}
		\includegraphics[width=\textwidth]{./images/bolt/car.png}
		\caption{BOLT car}
	\end{subfigure}
	\begin{subfigure}{0.18\textwidth}
		\includegraphics[width=\textwidth]{./images/bolt/headset.png}
		\caption{BOLT headset}
	\end{subfigure}
\end{figure}

\begin{figure}[H]
	\centering
	\begin{subfigure}{0.18\textwidth}
		\includegraphics[width=\textwidth]{./images/knn/plane.png}
		\caption{KNN plane}
	\end{subfigure}
	\begin{subfigure}{0.18\textwidth}
		\includegraphics[width=\textwidth]{./images/knn/helmet.png}
		\caption{KNN helmet}
	\end{subfigure}
	\begin{subfigure}{0.18\textwidth}
		\includegraphics[width=\textwidth]{./images/knn/cap.png}
		\caption{KNN cap}
	\end{subfigure}
	\begin{subfigure}{0.18\textwidth}
		\includegraphics[width=\textwidth]{./images/knn/car.png}
		\caption{KNN car}
	\end{subfigure}
	\begin{subfigure}{0.18\textwidth}
		\includegraphics[width=\textwidth]{./images/knn/headset.png}
		\caption{KNN headset}
	\end{subfigure}
\end{figure}

No filtering:

\begin{figure}[H]
	\centering
	\begin{subfigure}{0.18\textwidth}
		\includegraphics[width=\textwidth]{./images/octree/plane_4096.png}
		\caption{No filtering plane}
	\end{subfigure}
	\begin{subfigure}{0.18\textwidth}
		\includegraphics[width=\textwidth]{./images/octree/helmet_4096.png}
		\caption{No filtering helmet}
	\end{subfigure}
	\begin{subfigure}{0.18\textwidth}
		\includegraphics[width=\textwidth]{./images/octree/cap_4096.png}
		\caption{No filtering cap}
	\end{subfigure}
	\begin{subfigure}{0.18\textwidth}
		\includegraphics[width=\textwidth]{./images/octree/car_4096.png}
		\caption{No filtering car}
	\end{subfigure}
	\begin{subfigure}{0.18\textwidth}
		\includegraphics[width=\textwidth]{./images/octree/headset_4096.png}
		\caption{No filtering headset}
	\end{subfigure}
\end{figure}



\subsection{Comparison with MLS}

In this paper, the MLS implementation was written entirely in C++ using the Point Cloud Library (PCL) \cite{pcl}.
On average the execution time of MLS was 0.2 seconds, while our method took 0.5 seconds.
We also ran the experiments with a nearest neighbours value of 20, and polynomial order of 2. 
Below are qualitative results of our method compared to MLS.
The C++ implementation of MLS upsamples to a very large number of points, thus to compare we sampled the desired number of points from the MLS output to compare with ours.

The result of the raw MLS output without sampling points:

\begin{figure}[H]
	\centering
	\begin{subfigure}{0.18\textwidth}
		\includegraphics[width=\textwidth]{./images/mls/plane_all.png}
		\caption{MLS plane}
	\end{subfigure}
	\begin{subfigure}{0.18\textwidth}
		\includegraphics[width=\textwidth]{./images/mls/helmet_all.png}
		\caption{MLS helmet}
	\end{subfigure}
	\begin{subfigure}{0.18\textwidth}
		\includegraphics[width=\textwidth]{./images/mls/cap_all.png}
		\caption{MLS cap}
	\end{subfigure}
	\begin{subfigure}{0.18\textwidth}
		\includegraphics[width=\textwidth]{./images/mls/car_all.png}
		\caption{MLS car}
	\end{subfigure}
	\begin{subfigure}{0.18\textwidth}
		\includegraphics[width=\textwidth]{./images/mls/headset_all.png}
		\caption{MLS headset}
	\end{subfigure}
\end{figure}

The result of the MLS output with sampling points:

\begin{figure}[H]
	\centering
	\begin{subfigure}{0.18\textwidth}
		\includegraphics[width=\textwidth]{./images/mls/plane_4096.png}
		\caption{MLS plane}
	\end{subfigure}
	\begin{subfigure}{0.18\textwidth}
		\includegraphics[width=\textwidth]{./images/mls/helmet_4096.png}
		\caption{MLS helmet}
	\end{subfigure}
	\begin{subfigure}{0.18\textwidth}
		\includegraphics[width=\textwidth]{./images/mls/cap_4096.png}
		\caption{MLS cap}
	\end{subfigure}
	\begin{subfigure}{0.18\textwidth}
		\includegraphics[width=\textwidth]{./images/mls/car_4096.png}
		\caption{MLS car}
	\end{subfigure}
	\begin{subfigure}{0.18\textwidth}
		\includegraphics[width=\textwidth]{./images/mls/headset_4096.png}
		\caption{MLS headset}
	\end{subfigure}
\end{figure}

The BOLT results as a comparison:

\begin{figure}[H]
	\centering
	\begin{subfigure}{0.18\textwidth}
		\includegraphics[width=\textwidth]{./images/bolt/plane.png}
		\caption{BOLT plane}
	\end{subfigure}
	\begin{subfigure}{0.18\textwidth}
		\includegraphics[width=\textwidth]{./images/bolt/helmet.png}
		\caption{BOLT helmet}
	\end{subfigure}
	\begin{subfigure}{0.18\textwidth}
		\includegraphics[width=\textwidth]{./images/bolt/cap.png}
		\caption{BOLT cap}
	\end{subfigure}
	\begin{subfigure}{0.18\textwidth}
		\includegraphics[width=\textwidth]{./images/bolt/car.png}
		\caption{BOLT car}
	\end{subfigure}
	\begin{subfigure}{0.18\textwidth}
		\includegraphics[width=\textwidth]{./images/bolt/headset.png}
		\caption{BOLT headset}
	\end{subfigure}
\end{figure}

From a qualitative perspective, MLS does look better except for some cases like the car. It tends to preserve the roundness of objects better than BOLT.

\subsection{Comparison with Other Sampling Methods}

Here we compare our octree sampling method with random sampling, both cases using bilateral smoothing.

Octree sampling:

\begin{figure}[H]
	\centering
	\begin{subfigure}{0.18\textwidth}
		\includegraphics[width=\textwidth]{./images/bolt/plane.png}
		\caption{BOLT plane}
	\end{subfigure}
	\begin{subfigure}{0.18\textwidth}
		\includegraphics[width=\textwidth]{./images/bolt/helmet.png}
		\caption{BOLT helmet}
	\end{subfigure}
	\begin{subfigure}{0.18\textwidth}
		\includegraphics[width=\textwidth]{./images/bolt/cap.png}
		\caption{BOLT cap}
	\end{subfigure}
	\begin{subfigure}{0.18\textwidth}
		\includegraphics[width=\textwidth]{./images/bolt/car.png}
		\caption{BOLT car}
	\end{subfigure}
	\begin{subfigure}{0.18\textwidth}
		\includegraphics[width=\textwidth]{./images/bolt/headset.png}
		\caption{BOLT headset}
	\end{subfigure}
\end{figure}

Random sampling:

\begin{figure}[H]
	\centering
	\begin{subfigure}{0.18\textwidth}
		\includegraphics[width=\textwidth]{./images/random/plane.png}
		\caption{Random plane}
	\end{subfigure}
	\begin{subfigure}{0.18\textwidth}
		\includegraphics[width=\textwidth]{./images/random/helmet.png}
		\caption{Random helmet}
	\end{subfigure}
	\begin{subfigure}{0.18\textwidth}
		\includegraphics[width=\textwidth]{./images/random/cap.png}
		\caption{Random cap}
	\end{subfigure}
	\begin{subfigure}{0.18\textwidth}
		\includegraphics[width=\textwidth]{./images/random/car.png}
		\caption{Random car}
	\end{subfigure}
	\begin{subfigure}{0.18\textwidth}
		\includegraphics[width=\textwidth]{./images/random/headset.png}
		\caption{Random headset}
	\end{subfigure}
\end{figure}

\subsection{Deep Learning Comparison}

In this section we show the outputs of the deep learning method used in this paper, PU-GCN \cite{PU-GCN}. 
In this case we used the pretrained model provided by the authors of the paper.
We also instead of starting from 1024 points, we started from 256 points and upsampled to 1024 points.
Also instead of shapenet, we used the same dataset as in the paper, the PU1K dataset.

The deep learning method results:

\begin{figure}[H]
	\centering 
	\begin{subfigure}{0.45\textwidth}
		\includegraphics[width=\textwidth]{./images/pu-gcn/eight.png}
		\caption{Eight}
	\end{subfigure}
	\begin{subfigure}{0.45\textwidth}
		\includegraphics[width=\textwidth]{./images/pu-gcn/elephant.png}
		\caption{Elephant}
	\end{subfigure}
\end{figure}
\begin{figure}[H]
	\begin{subfigure}{0.45\textwidth}
		\includegraphics[width=\textwidth]{./images/pu-gcn/elk.png}
		\caption{Elk}
	\end{subfigure}
	\begin{subfigure}{0.45\textwidth}
		\includegraphics[width=\textwidth]{./images/pu-gcn/fandisk.png}
		\caption{Fandisk}
	\end{subfigure}
\end{figure}
\begin{figure}[H]
	\begin{subfigure}{0.45\textwidth}
		\includegraphics[width=\textwidth]{./images/pu-gcn/genus3.png}
		\caption{Genus3}
	\end{subfigure}
\end{figure}

\begin{figure}[H]
	\centering
	\begin{subfigure}{0.18\textwidth}
		\includegraphics[width=\textwidth]{./images/bolt/eight.png}
		\caption{BOLT eight}
	\end{subfigure}
	\begin{subfigure}{0.18\textwidth}
		\includegraphics[width=\textwidth]{./images/bolt/elephant.png}
		\caption{BOLT elephant}
	\end{subfigure}
	\begin{subfigure}{0.18\textwidth}
		\includegraphics[width=\textwidth]{./images/bolt/elk.png}
		\caption{BOLT elk}
	\end{subfigure}
	\begin{subfigure}{0.18\textwidth}
		\includegraphics[width=\textwidth]{./images/bolt/fandisk.png}
		\caption{BOLT fandisk}
	\end{subfigure}
	\begin{subfigure}{0.18\textwidth}
		\includegraphics[width=\textwidth]{./images/bolt/genus3.png}
		\caption{BOLT genus3}
	\end{subfigure}
\end{figure}


